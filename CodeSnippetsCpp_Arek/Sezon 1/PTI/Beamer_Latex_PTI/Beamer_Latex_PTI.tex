%%%%%%%%%%%%%%%%%%%%%%%%%%%%% Nie wiem jak zmodyfikować tę stopkę

\documentclass{beamer}
\usepackage[polish]{babel} % język polski
\usepackage[MeX]{polski} % język polski
\usepackage[utf8]{inputenc} % język polski, latin2 lub cp1250
\usetheme{Madrid}
\useoutertheme{smoothbars} % modyfikacja nagłowku każdego slajdu
\useinnertheme{rounded} % do modyfikacji takich bzdur jak na przykład liczby w kółeczkach przy enumeracji
%\setbeamercovered{transparent} % transparent zostawia lekko widoczne inne czesci slajdu, pomiedzy przejsciami
\usecolortheme{whale}



\title{Prezentacja PTI}
\subtitle {Czyli droga przez mękę}
\author{Arkadiusz Marks}
\institute{AiR I}
\date{6 stycznia 2017}
 
 
 
\begin{document}

\begin{frame}
	\maketitle
\end{frame}

\begin{frame} %{Spis Treści} opcjonalny naglówek slajdu
	\tableofcontents
\end{frame}


\section{Sekcja pełna list} 


\begin{frame}
	%\frametitle{Sample frame title}
	Przedmioty które lubię:
	\begin{enumerate}
		\item<1-> PTI
		\item<2-> Język Polski
		\item<3->	Widelec
	\end{enumerate}
\end{frame}

\begin{frame}
	%\frametitle{Sample frame title}
	Przedmioty których nie lubię:
	\begin{enumerate}
		\item<1-1> PTI
		\item<2-2> Język Polski
		\item<3-3>	Widelec
	\end{enumerate}
\end{frame}
 
\section{Sekcja pełna bloczków}


\begin{frame}
	\onslide<1-2>
	{
		\begin{block}{Bloczek Optymistyczny}
		PTI to przedmiot łatwy i przyjemny, a także \alert{bardzo rozwijający}
		\end{block}
	}
	\onslide<2-2>
	{
		\begin{block}{Bloczek Pesymistyczny}
		 PTI to przedmiot łatwy i przyjemny, a także \alert{szalenie trudny do zaliczenia}
		\end{block}
	}
	\onslide<3-3>
	{
		\setbeamercolor{block body}{use=structure,fg=black,bg=yellow}
		\begin{block}{}
		 A to żółty bloczek
		\end{block}
	}
\end{frame}
\end{document}